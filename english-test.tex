\documentclass[12pt]{paper}
\usepackage[T1]{fontenc}
\usepackage{indentfirst}
\usepackage{textcomp}
\renewcommand{\rmdefault}{ptm}
\usepackage[scaled=1.1]{helvet}
\usepackage{mtpro2}
\usepackage{amsfonts, amsmath, amsthm, amssymb}
\usepackage[version=4]{mhchem}
\usepackage[margin=1in]{geometry}
\numberwithin{equation}{section}
\newtheorem{question}{Question}
\theoremstyle{definition}
\newtheorem*{answer}{Answer}
\renewcommand\thesection{Passage \Roman{section}}
\numberwithin{equation}{section}
\usepackage{fancyhdr}
\pagestyle{fancy}
\fancyhf{}
% \usepackage{titlesec}
% \titleformat{\section}{\LARGE\bfseries}{Chapter \thesection.}{1em}{}

% % % % % % % % % %
\title{Pre IELTS Level / Listening \& Reading\\
Test 01 (Unit 2, 4, 6, 8)}
\author{gtdmmr-academy$^\dagger$}
% \date{\today}
\begin{document}
\maketitle
\vfill
$^\dagger$\textit{ngdnhtien.com}
\newpage

\section{The Motor Car}
\textbf{A.} There are now over 700 million motor vehicles in the world - and the number is rising by more than 40 million each year. The average distance driven by car users is growing too - from 8 km a day per person in western Europe in 1965 to 25 km a day in 1995. This dependence on motor vehicles has given rise to major problems, including environmental pollution, reduction in oil resources, traffic congestion and safety.

\textbf{B.} While emissions from new cars are far less harmful than they used to be, city streets and motorways are becoming more crowded than ever, often with older trucks, buses and taxis, which emit excessive levels of smoke and fumes. This concentration of vehicles makes air quality in urban areas unpleasant and sometimes dangerous to breathe. Even Moscow has joined the list of capitals afflicted by congestion and traffic fumes. In Mexico City, vehicle pollution is a major health hazard.

\textbf{C.} Until a hundred years ago, most journeys were in the 20 km range, the distance conveniently accessible by horse. Heavy freight could only be carried by water or rail. The invention of the motor vehicle brought personal mobility to the masses and made rapid freight delivery possible over a much wider area. Today about 90 per cent of inland freight in the United Kingdom is carried by road. Clearly the world cannot revert to the horse-drawn wagon. Can it avoid being locked into congested and polluting ways of transporting people and goods?

\textbf{D.} In Europe most cities are still designed for the old modes of transport. Adaptation to the motor car has involved adding ring roads, one-way systems and parking lots. In the United States, more land is assigned to car use than to housing. Urban sprawl means that life without a car is next to impossible. Mass use of motor vehicles has also killed or injured millions of people. Other social effects have been blamed on the car such as alienation and aggressive human behaviour.

\textbf{E.} A 1993 study by the European Federation for Transport and Environment found that car transport is seven times as costly as rail travel in terms of the external social costs it entails such as congestion, accidents, pollution, loss of cropland and natural habitats, reduction in oil resources, and so on. Yet cars easily surpass trains or buses as a flexible and convenient mode of personal transport. It is unrealistic to expect people to give up private cars in favour of mass transit.

\textbf{F.} Technical solutions can reduce the pollution problem and increase the fuel efficiency of engines. But fuel consumption and exhaust emissions depend on which cars are preferred by customers and how they are driven. Many people buy larger cars than they need for daily purposes or waste fuel by driving aggressively. Besides, global car use is increasing at a faster rate than the improvement in emissions and fuel efficiency which technology is now making possible.

\textbf{G.} One solution that has been put forward is the long-term solution of designing cities and neighbourhoods so that car journeys are not necessary - all essential services being located within walking distance or easily accessible by public transport. Not only would this save energy and cut carbon dioxide emissions, it would also enhance the quality of community life, putting the emphasis on people instead of cars. Good local government is already bringing this about in some places. But few democratic communities are blessed with the vision - and the capital - to make such profound changes in modern lifestyles.

\textbf{H.} A more likely scenario seems to be a combination of mass transit systems for travel into and around cities, with small 'low emission' cars for urban use and larger hybrid or learn burn cars for use elsewhere. Electronically tolled highways might be used to ensure that drivers pay charges geared (\textit{adjusted}) to actual road use. Better integration of transport systems is also highly desirable and made more feasible by modern computers. But these are solutions for countries which can afford them. In most developing countries, old cars and old technologies continue to predominate.

\begin{question}
    Reading Passage 1 has eight paragraphs (A-H). Which paragraphs concentrate on the following information? Write the appropriate letters (A-H) next to the questions.
    \begin{enumerate}
        \item[1.1] a comparison of past and present transportation methods.
        \item[1.2] how driving habits contribute to road problems.
        \item[1.3] the relative merits (advantages) of cars and public transport.
        \item[1.4] the writer's own prediction of future solutions.
        \item[1.5] the increasing use of motor vehicles.
        \item[1.6] the impact of the car on city development.
    \end{enumerate}
\end{question}

\begin{question}
    Do the following statements agree with the information given in Reading Passage 1? Write next to the questions
    \begin{center}
        \begin{tabular}{ l l }
            \textbf{YES} & if the statement agrees with the information \\
            \textbf{NO} & if the statement contradicts the information \\
            \textbf{NOT GIVEN} & if there is no information on this in the passage   
        \end{tabular}
    \end{center}
    \begin{enumerate}
        \item[2.1] Vehicle pollution is worse in European cities than anywhere else.
        \item[2.2] Transport by horse would be a useful alternative to motor vehicles.
        \item[2.3] Nowadays freight is not carried by water in the United Kingdom.
        \item[2.4] Most European cities were not designed for motor vehicles.
        \item[2.5] Technology alone cannot solve the problem of vehicle pollution.
        \item[2.6] People's choice of car and attitude to driving is a factor in the pollution problem.
        \item[2.7] Redesigning cities would be a short-term solution.
    \end{enumerate}
\end{question}

\section{Reading The Screen}

The debate surrounding literacy is one of the most charged in education. On the one hand there is an army of people convinced that traditional skills of reading and writing are declining. On the other, a host of progressives protest that literacy is much more complicated than a simple technical mastery of reading and writing. This second position is supported by most of the relevant academic work over the past 20 years. These studies argue that literacy can only be understood in its social and technical context. In Renaissance England, for example, many more people could read than could write, and within reading there was a distinction (difference) between those who could read print and those who could manage the more difficult task of reading manuscript. An understanding of these earlier periods helps us understand today's 'crisis in literacy' debate.

There does seem to be evidence that there has been an overall decline in some aspects of reading and writing - you only need to compare the tabloid newspapers of today with those of 50 years ago to see a clear decrease in vocabulary and simplification of syntax. But the picture is not uniform and doesn't readily demonstrate the simple distinction between literate and illiterate which had been considered adequate (enough) since the middle of the 19th century.

While reading a certain amount of writing is as crucial as it has ever been in industrial societies, it is doubtful whether a fully extended grasp of either is as necessary as it was 30 or 40 years ago. While print retains much of its authority as a source of topical information, television has increasingly usurped (stole) this role. The ability to write fluent letters has been undermined by the telephone and research suggests that for many people the only use for writing, outside formal education, is the compilation of shopping lists.

The decision of some car manufacturers to issue their instructions to mechanics as a video pack rather than as a handbook might be taken to spell the end of any automatic link between industrialisation and literacy. On the other hand, it is also the case that ever-increasing numbers of people make their living out of writing, which is better rewarded than ever before. Schools are generally seen as institutions where the book rules - film, television and recorded sound have almost no place; but it is not clear that this opposition is appropriate. While you may not need to read and write to watch television, you certainly need to be able to read and write in order to make programmes.

Those who work in the new media are anything but illiterate. The traditional oppositions between old and new media are inadequate (not enough) for understanding the world which a young child now encounters (faces). The computer has re-established a central place for the written word on the screen, which used to be entirely devoted to the image. There is even anecdotal (transferred by mouths) evidence that children are mastering reading and writing in order to get on to the Internet. There is no reason why the new and old media cannot be integrated in schools to provide the skills to become economically productive and politically enfranchised (free).

Nevertheless, there is a crisis in literacy and it would be foolish to ignore it. To understand that literacy may be declining because it is less central to some aspects of everyday life is not the same as acquiescing in this state of affairs. The production of school work with the new technologies could be a significant stimulus to literacy. How should these new technologies be introduced into the schools? It isn't enough to call for computers, camcorders and edit suites in every classroom; unless they are properly integrated into the educational culture, they will stand unused. Evidence suggests that this is the fate of most information technology used in the classroom. Similarly, although media studies are now part of the national curriculum, and more and more students are now clamouring (wanting) to take these course, teachers remain uncertain about both methods and aims in this area.

This is not the fault of the teachers. The entertainment and information industries must be drawn into a debate with the educational institutions to determine how best to blend these new technologies into the classroom.

Many people in our era are drawn to the pessimistic (not optimistic) view that the new media are destroying old skills and eroding critical judgement. It may be true that past generations were more literate but - taking the pre-19th century meaning of the term - this was true of only a small section of the population. The word literacy is a 19th-century coinage (term) to describe the divorce of reading and writing from a full knowledge of literature. The education reforms of the 19th century produced reading and writing as skills separable from full participation in the cultural heritage.

The new media now point not only to a futuristic cyber-economy, they also make our cultural past available to the whole nation. Most children's access to these treasures is initially through television. It is doubtful whether our literary heritage has ever been available to or sought out by more than about 5 per cent of the population; it has certainly not been available to more than 10 per cent. But the new media joined to the old, through the public service tradition of British broadcasting, now makes our literary tradition available to all.

\begin{question}
    Choose the appropriate letters A-D and circle them
    \begin{enumerate}
        \item[3.1] When discussing the debate on literacy in education, the writer notes that
        \begin{itemize}
            \item[A.] children cannot read and write as well as they used to.
            \item[B.] academic work has improved over the last 20 years.
            \item[C.] there is evidence that literacy is related to external factors.
            \item[D.] there are opposing arguments that are equally convincing.
        \end{itemize}
        \item[3.2] In the 4th paragraph, the writer's main point is that
        \begin{itemize}
            \item[A.] the printed word is both gaining and losing power.
            \item[B.] all inventions bring disadvantages as well as benefits.
            \item[C.] those who work in manual jobs no longer need to read.
            \item[D.] the media offers the best careers for those who like writing.
        \end{itemize}
        \item[3.3] According to the writer, the main problem that schools face today is
        \begin{itemize}
            \item[A.] how best to teach the skills of reading and writing.
            \item[B.] how best to incorporate technology into classroom teaching.
            \item[C.] finding the means to purchase technological equipment.
            \item[D.] managing the widely differing levels of literacy amongst pupils.
        \end{itemize}
        \item[3.4] At the end of the article, the writer is suggesting that
        \begin{itemize}
            \item[A.] literature and culture cannot be divorced.
            \item[B.] the term 'literacy' has not been very useful. 
            \item[C.] 10 per cent of the population never read literature. 
            \item[D.] our exposure to cultural information is likely to increase. 
        \end{itemize}
    \end{enumerate}
\end{question}
\begin{question}
    Do the following statements agree with the information given in Reading Passage 2? Write next to the questions
    \begin{center}
        \begin{tabular}{ l l }
            \textbf{YES} & if the statement agrees with the information \\
            \textbf{NO} & if the statement contradicts the information \\
            \textbf{NOT GIVEN} & if there is no information on this in the passage   
        \end{tabular}
    \end{center}
    \begin{enumerate}
        \item[4.1] It is not as easy to analyse literacy levels as it used to be.
        \item[4.2] Our literacy skills need to be as highly developed as they were in the past.
        \item[4.3] Illiteracy is on the increase.
        \item[4.4] Professional writers earn relatively more than they used to.
        \item[4.5] A good literacy level is important for those who work in television.
        \item[4.6] Computers are having a negative impact on literacy in schools.
    \end{enumerate}
\end{question}
\begin{question}
    Complete the sentences below with words taken from Reading Passage 2. Use \textit{NO MORE THAN THREE WORDS} for each answer.\par 
    \begin{enumerate}
        \item[5.1] In Renaissance England, the best readers were those able to read $\dots\dots\dots$.
        \item[5.2] The writer uses the example of $\dots\dots\dots$ to illustrate the general fall in certain areas of literacy.
        \item[5.3] It has been shown that after leaving school, the only things that a lot of people write are $\dots\dots\dots$.
    \end{enumerate}
\end{question}
\centering -- END OF THE TEST --
\end{document}