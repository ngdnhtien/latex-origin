\documentclass{beamer}
\usepackage[utf8]{vietnam}
\usepackage{physics}
\usepackage{amsthm}
\usepackage{braket}
\newtheorem{remark}{Remark}
\newcommand{\dcp}{\delta_{CP}}
\renewcommand\bra[1]{{\langle{#1}|}}
\makeatletter
\renewcommand\ket[1]{%
  \@ifnextchar\bra{\k@t{#1}\!}{\k@t{#1}}%
}
\newcommand\k@t[1]{{|{#1}\rangle}}
\DeclareMathAlphabet\mathbfcal{OMS}{cmsy}{b}{n}
\makeatother
\usetheme{Pittsburgh}
\usecolortheme{default}
\AtBeginSection[]
{
  \begin{frame}
    \frametitle{Chương trình nghị/hoạt sự}
    \tableofcontents[currentsection]
  \end{frame}
}

%Information to be included in the title page:
\title[HPC] %optional
{SINH HOẠT THƯỜNG KỲ\\
Tuần 8 (7--14/4)}

\author[] % (optional, for multiple authors)
{Nguyễn Đình Tiến}

\institute[] % (optional)
{
Trưởng Ban Chuyên môn\\
Câu lạc bộ Vật lý (HPC)
}

\date{}


\begin{document}

\frame{\titlepage}
\tableofcontents
\section{Không gian làm việc online}
\begin{frame}
    \frametitle{Tạm biệt Slack. Xin chào Discord}
    Sau một thời gian thử nghiệm nội bộ Ban Chuyên môn với kế hoạch sử dụng Slack để làm việc, mình nhận thấy Slack có một vài nhược điểm
            \begin{itemize}
                \item Không cho phép họp trực tuyến từ 02 người dùng trở lên (phiên bản \textit{pagoda});
                \item Không cho phép tạo kênh liên lạc riêng với external agent\footnote{người dùng bên ngoài tổ chức};
                \item Giao diện chưa thân thiện, cần thời gian để làm quen.
            \end{itemize}
            Do vậy, chúng ta sẽ chuyển đổi toàn bộ hoạt động đang diễn ra trên Slack sang máy chủ Discord \textit{HUS Physics Club}. 
\end{frame}
\begin{frame}
    \frametitle{Tạm biệt Slack. Xin chào Discord}
    \begin{columns}
        \begin{column}{0.6\textwidth}
            Việc chuyển đổi từ Messenger sang Discord có những ưu điểm bao gồm:
            \begin{itemize}
                \item Tăng độ trong suốt (transparency) của toàn bộ CLB;
                \item Tăng hiệu quả làm việc thông qua các kênh liên lạc đa kênh;
                \item Hỗ trợ sắp xếp và lưu trữ dữ liệu, phục vụ truy xuất tốt hơn;
                \item Có khả năng làm việc với external agent trong các trường hợp đặc biệt, ví dụ: Cách mạng $A^{+}$.
            \end{itemize}
        \end{column}
        \begin{column}{0.4\textwidth}
            \begin{figure}
                \centering 
                \includegraphics[scale=0.4]{discord.png}
                \caption{Quét mã QR trên để gia nhập máy chủ HUS Physics Club trên Discord.}
            \end{figure}
        \end{column}
    \end{columns}
\end{frame}
\section{Cập nhật tiến độ}
\begin{frame}
    \frametitle{Cập nhật tiến độ (7--14/4)}
    Các dự án đã giao và đang thực hiện bao gồm
    \begin{enumerate}
        \item Workshop "\textit{Vật lý trong cuộc sống}"\\ Diễn giả khách mời: Phúc Đạt;
        \item Dự án \textit{Bóng thám không}\\
        Leader: Nam Sơn (Ban Chuyên môn);
        \item Kế hoạch dã ngoại và quay phim ngắn\\
        Leader: Thuỳ Linh (Ban Chuyên môn);
        \item HPCTalk lần 4, "Neutron trong trường hấp dẫn của Trái Đất"\\
        Diễn giả: Thế Nam (Ban Chuyên môn).
    \end{enumerate}
\end{frame}
\section{Kết luận \& Kế hoạch tuần tới}
\begin{frame}
    \frametitle{Kết luận \& Kế hoạch Tuần 9}

    

\end{frame}

\section{Lab tour Khoa Vật lý}
\begin{frame}
    \frametitle{Lab tour Khoa Vật lý}

    

\end{frame}
\end{document}